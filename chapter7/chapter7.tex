\documentclass[12pt]{article}

\usepackage{tikz}
\usepackage{cancel}
\usepackage{amsthm}
\usepackage{amssymb}
\usepackage{amsmath}
\usepackage{latexsym}
\usepackage{mathtools}
\usepackage{stackengine}

\usepackage{fancyhdr}

\usepackage{float}
\usepackage{xcolor}
\usepackage{graphicx}
\usepackage{pagecolor}

\usepackage{soul}

\usepackage[
top=2.50cm,
bottom=2.50cm,
left=2cm,
right=2cm,
marginparsep=0pt,
marginparwidth=0pt]{geometry}

\parindent 0in
\parskip 12pt

\definecolor{Ivory Paper}{HTML}{F7F0DF}

\pagecolor{Ivory Paper}

\newcommand{\floor}[1]{\left\lfloor #1 \right\rfloor}
\newcommand{\ceil}[1]{\left\lceil #1 \right\rceil}
\newcommand{\round}[1]{\left\lfloor #1 \right\rceil}
\newcommand{\abs}[1]{\left\lvert #1 \right\rvert}
\newcommand{\sqref}[1]{[\ref{#1}]}

\newcommand{\mH}{\char"0126}
\newcommand{\mh}{\char"0127}
\newcommand{\mG}{\char"0120}
\newcommand{\mg}{\char"0121}
\newcommand{\mC}{\char"010A}
\newcommand{\mc}{\char"010B}
\newcommand{\mZ}{\char"017B}
\newcommand{\mz}{\char"017C}

\newcommand{\defeq}{\stackrel{\text{def}}{=}}
\newcommand{\msteq}{\stackrel{\text{mst}}{=}}
\newcommand{\sqemptyset}{\ensurestackMath{\stackinset{c}{}{c}{}{/}{\square}}}
\newcommand{\sqin}{%
  \mathrel{\vphantom{\sqsubset}\text{%
    \mathsurround=0pt
    \ooalign{$\sqsubset$\cr$-$\cr}%
  }}%
}


% TODO: Reorder
\DeclareMathOperator{\jseq}{seq}
\DeclareMathOperator{\jdom}{dom}
\DeclareMathOperator{\jupto}{upto}
\DeclareMathOperator{\jundefined}{undefined}
\DeclareMathOperator{\jcons}{cons}
\DeclareMathOperator{\jhead}{head}
\DeclareMathOperator{\jtail}{tail}
\DeclareMathOperator{\jlen}{len}
\DeclareMathOperator{\jtotal}{total}
\DeclareMathOperator{\jfunctional}{functional}
\DeclareMathOperator{\jinjective}{injective}
\DeclareMathOperator{\jsurjective}{surjective}

\theoremstyle{plain}
\newtheorem{thm}{Theorem}
\newtheorem{cor}{Corollary}
\newtheorem{lma}{Lemma}
\newtheorem{prop}{Proposition}
\newtheorem{conj}{Conjecture}
\newtheorem{defn}{Definition}

\setlength{\headheight}{15pt}
\pagestyle{fancy}
\renewcommand{\headrulewidth}{0pt}
\lhead{J. Scerri}
\chead{Noti g\mh al \st{Qabel} Waqt l-E\.zami}
\rhead{\thepage}

% TITLE

\title{Discrete 2 (Chapter 7)\\
\vspace{0.75em}\textbf{Noti g\mh al \st{Qabel} Waqt l-E\.zami}}
\author {{\textbf{Juan Scerri}}\\
B.Sc. (Hons)(Melit.) Computing Science and Mathematics (Second Year)}

\begin{document}

\maketitle % Print the title page

\thispagestyle{empty} % Suppress headers and footers on the title page

\raggedright

\section{Relations}

\textbf{Totality:} $r \in X \leftrightarrow Y$ is said to be
total iff $\jtotal(r) \defeq \forall\, x \in X, \exists\, y \in
Y \cdot (x, y) \in r$ it true.

\textbf{Functionality:} $r \in X \leftrightarrow Y$ is said to
be functional iff $\jfunctional(r) \defeq\forall\, y_1, y_2 \in
Y, \forall\, x \in X \cdot (x, y_1) \in r \land (x, y_2)
\Rightarrow y_1 = y_2$ is true.

\textbf{Surjectivity:} $r \in X \leftrightarrow Y$ is said to be
surjective iff $\jsurjective(r) \defeq\forall\, y \in Y,
\exists\, x \in X \cdot (x, y) \in r$ is true.

\textbf{Injectivity:} $r \in X \leftrightarrow Y$ is said to be
injective iff $\jinjective(r) \defeq \forall\, x_1, x_2 \in X,
\forall\, y \in Y \cdot (x_1, y) \in r \land (x_2, y)
\Rightarrow x_1 = x_2$ is true.

\textbf{Set of Total and Functional Relations:} $X \rightarrow Y
\defeq \{r \in X \leftrightarrow Y : \jtotal(r)
\land \jfunctional(r) \}$

\section{Multisets}

\subsection{Definitions}

\textbf{Set of Multisets Over $X$:} $\mathbb{M}\,X \defeq X
\rightarrow \mathbb{N}$.

\textbf{Empty Multiset:} $\sqemptyset \defeq \lambda\, x \in X
\cdot 0$.

\textbf{Multiset Element Of:} $x \sqin T \defeq T(x) > 0$

\textbf{Multiset Subset:} $T \sqsubseteq S \defeq \forall\, x
\in X \cdot T(x) \leq S(x)$.

\textbf{Multiset Equality:} $T \msteq S \defeq \forall\, x \in X
\cdot T(x) = S(x)$.

\textbf{Multiset Union:} $T \sqcup S \defeq \lambda\, x \in X
\cdot T(x) + S(x)$.

\textbf{Multiset Intersection:} $T \sqcap S \defeq \lambda\, x
\in X \cdot \min(\{T(x), S(x)\})$.

\subsection{Theorems}

\textbf{Theorem 7.4:} $T \sqcup S \msteq S \sqcup T$.

\textbf{Theorem 7.6:} $T \sqcap S \sqsubseteq T \sqcup S$.

\subsection{Exercises}

\textbf{Exercise 7.1:} $T \sqsubseteq T$ (reflexive) and $T
\sqsubseteq R \land R \sqsubseteq S \Rightarrow T \sqsubseteq S$
(transitive).

\textbf{Exercise 7.2:} $\sqemptyset \sqcap T \msteq T \sqcap
\sqemptyset \msteq \sqemptyset$ and $\sqemptyset \sqcup T \msteq
T \sqcup \sqemptyset \msteq T$.

\textbf{Exercise 7.3:} $T \sqcap T \msteq T$ (idempotent), $T
\sqcap S \msteq S \sqcap T$ (commutative) and $(T \sqcap S)
\sqcap R \msteq T \sqcap (S \sqcap R)$ (associative).

\textbf{Exercise 7.4:} $(T \sqcup S)
\sqcup R \msteq T \sqcup (S \sqcup R)$ (associative).

\textbf{Exercise 7.5:} $T - S \defeq \lambda\, x \in X \cdot
\max(\{0, T(x) - S(x)\})$ (multiset difference), $T - T \msteq
\sqemptyset$ and $T - \sqemptyset \msteq T$.

\section{Sequences}

\textbf{Finite Subsets of $\mathbb{N}$:} $\jupto(N)
\defeq \{n \in \mathbb{N} : n < N\}\ (= \{0, 1, \ldots, N - 1\})$.

\textbf{Set of Sequences Over $X$:} $\jseq\, X = \{s_X \in
\mathbb{N} \rightarrow X : \exists\, N \in \mathbb{N} \cdot
\jdom(s_X) = \jupto(N)\}$.

\textbf{Empty Sequence:} $\langle \rangle \defeq \lambda\, n \in
\mathbb{N} \cdot \jundefined$.

\textbf{Head of a Sequence:} $\jhead(s_X) \defeq s_X(0)$.

\textbf{Tail of a Sequence:} $\jtail(s_X) \defeq \lambda\, n \in
\mathbb{N} \cdot s_X(n + 1)$.

\textbf{Cons Operator:}
$$
\jcons(x, s_X) \defeq \lambda\, n \in \mathbb{N}
\cdot
\begin{cases}
    x          & \text{if}\ n = 0\\
    s_X(n - 1) & \text{otherwise}
\end{cases}.
$$

\textbf{Length of a Sequence:}

\begin{align*}
\jlen(s_X) &\defeq
\begin{cases}
    0                    & \text{if}\ n = 0\\
    1 + \max(\jdom(s_X)) & \text{otherwise}
\end{cases},\ \text{(Non-recursive)}.\\\\
    &\defeq
\begin{cases}
    0                   & \text{if}\ n = 0\\
    1 + \jlen(\jtail(s_X)) & \text{otherwise}
\end{cases},\ \text{(Recursive)}.
\end{align*}
$$
\jlen(s_X) \defeq
\begin{cases}
    0                   & \text{if}\ n = 0\\
    1 + \max(\jdom(s_X)) & \text{otherwise}
\end{cases},\ \text{(Non-recursive)}.
$$
$$
\jlen(s_X) \defeq
\begin{cases}
    0                   & \text{if}\ n = 0\\
    1 + \jlen(\jtail(s_X)) & \text{otherwise}
\end{cases},\ \text{(Recursive)}.
$$

\section{Graph Theory}


\end{document}
